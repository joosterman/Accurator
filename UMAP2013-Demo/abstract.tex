\begin{abstract}
A digital collection that can be accessed online, searched and linked to other collections is an important focus for many Cultural Heritage institutions. Diversity and depth of the topics in a collection make experts outside the institution indispensable to acquire qualitative annotations to support these actions. We define the concept of nichesourcing and present the challenges in the process of obtaining qualitative annotations from persons in these niches. Our assumption is that if this process is personalized we get better annotations from the experts. Our main contribution is a framework for nichesourcing, called Accurator, that allows to realize and evaluate strategies and applications for personalized nichesourcing.\keywords{cultural heritage, nichesourcing, annotation framework, qualitative annotations}
\end{abstract}