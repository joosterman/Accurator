\section{Introduction}\label{introduction}
For the enrichment of cultural heritage collections, the acquisition of qualitative annotations is a significant effort for museums and heritage institutions. In our research we approach the challenge to acquire qualitative annotations from communities in the crowd of people, i.e. people that are external to the institution. For this objective, we turn to personalised niche sourcing, where we aim to identify the niche communities and the ways to adapt the annotation task to them.

In this paper, we shortly describe the motivation behind the approach and the project in which the investigations take place, we present the four main research challenges that  drive the detailed investigations, and we present the main aspects of the implementation.
